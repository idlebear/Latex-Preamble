

% \newif\ifproofing\proofingtrue
% \newif\ifproofing\proofingfalse

% \ifproofing
%%%%%%%%%%%%%%%%%%%%%%%%%%%%%%%  proofing... %%%%%%%%%%%%%%%%%%%%%%%%%%%%%%%%%%%
\documentclass{article}
\usepackage[margin=1in]{geometry}
\usepackage{setspace}
\doublespacing
%%%%%%%%%%%%%%%%%%%%%%%%%%%%%%%  proofing... %%%%%%%%%%%%%%%%%%%%%%%%%%%%%%%%%%%
% \else
% %%%%%%%%%%%%%%%%%%%%%%%%% comment out for proofing... %%%%%%%%%%%%%%%%%%%%%%%%%%
% \documentclass[journal]{IEEEtran}
% \IEEEoverridecommandlockouts                              % This command is only needed if 
%                                                           % you want to use the \thanks command
% % \overrideIEEEmargins                                      % Needed to meet printer requirements.
% %%%%%%%%%%%%%%%%%%%%%%%%% comment out for proofing... %%%%%%%%%%%%%%%%%%%%%%%%%%
% \fi

\usepackage{preamble}
\usepackage{lipsum}

\title{\LARGE \bf
Title for Awesome Paper}


\urldef{\real}\url{totally.a.real.person@uwaterloo.ca}

\author{Real Person
\thanks{This research is partially supported by your local producer of recreational beverages. }
\thanks{The authors are with the Department of Electrical and Computer Engineering, University of Waterloo, Waterloo ON, N2L 3G1 Canada (\real)
    }
}


\begin{document}


\newpage
% \maketitle
% \thispagestyle{empty}
% \pagestyle{empty}

\section{Breaking Up Equations}
\theorem{Equation can be Broken}

When working with equations, there is a way to insert text comments and maintain alignment~\cite{wang2023cooperative}. 

\flag{This is an important note!} 
and some stuff after

\begin{proof}

It is occasionally useful to insert a longer comment that explains a step.  However, if you break the equation into separate alignment sections, then the alignment is lost:
 \begin{align}
    b_{(s,m),t+1}\left(\begin{matrix}
        s^a = i, \\ m^a = k
    \end{matrix} \right) &= \prb( s_{t+1}^a = i, m_{t+1}^a = k | o_{t+1}, s_{t}^e, u_{t}, s_t^a, m_t^a, b_{(s,m),t} ) \\
    &= \frac{\prb( o_{t+1} | s_{t+1}^a = i, m_{t+1}^a = k, s_{i,t}^e, u_{t},  s_t^a, m_t^a, b_{t} ) \prb( s_{t+1}^a = i, m_{t+1}^a = k |  s_{i,t}^e, u_{t},  s_t^a, m_t^a, b_{t} )}{\prb(o_{t+1} | s_{i,t}^e, u_{t}, s_t^a, m_t^a, b_{t}   ) }     
 \end{align}

Comment explaining why the previous line is way to complicated and can be easily simplified

\begin{align}
     &\propto \prb( o_{t+1} |  s_{t+1}^a = i, m_{t+1}^a = k, s_t^e, u_{t}, s_t^a, m_t^a, b_{t} ) \prb( s_{t+1}^a = i, m_{t+1}^a = k |  s_t^e, u_{t}, s_{t}^a, m_{t}^a, b_{(s,m),t} ).  \\
    &\propto \sum_{j=1}^{|S|} \left[ \prb( o_{t+1} |  s_{t+1}^a = i, m_{t+1}^a = k, s_t^e, u_{t}, s_t^a, b_{(s,m),t} ) \right. \nonumber \\ 
    &~~~~~~~~~~~~~~~~\left. \prb( s_{t+1}^a = i, m_{t+1}^a = k | s_{t}^a=j, u_{t}, m_t^a = k ) \prb( s_{t}^a = j, m_{t}^a = k |  b_{(s,m),t} ) \right],   \\
    &\propto  \sum_{j=1}^{|S|} O(o_{t+1},s_{t+1}^a=i,s_t^a=j,m_t=k) T_s(s_{t+1}^a=i, s_{t}^a=j, m_t ) b_{(s,m),t}(s_t^a = j, m_{t}^a = k).
\end{align}


\newpage

Dropping in a \textbackslash text\{ ... \} block is worse:

\begin{align}
     b_{(s,m),t+1}\left(\begin{matrix}
        s^a = i, \\ m^a = k
    \end{matrix} \right) &= \prb( s_{t+1}^a = i, m_{t+1}^a = k | o_{t+1}, s_{t}^e, u_{t}, s_t^a, m_t^a, b_{(s,m),t} ) \\
    &= \frac{\prb( o_{t+1} | s_{t+1}^a = i, m_{t+1}^a = k, s_{i,t}^e, u_{t},  s_t^a, m_t^a, b_{t} ) \prb( s_{t+1}^a = i, m_{t+1}^a = k |  s_{i,t}^e, u_{t},  s_t^a, m_t^a, b_{t} )}{\prb(o_{t+1} | s_{i,t}^e, u_{t}, s_t^a, m_t^a, b_{t}   ) } \\
    \text{ Comment explaining why the previous line is way to complicated and can be easily simplified } \\
          &\propto \prb( o_{t+1} |  s_{t+1}^a = i, m_{t+1}^a = k, s_t^e, u_{t}, s_t^a, m_t^a, b_{t} ) \prb( s_{t+1}^a = i, m_{t+1}^a = k |  s_t^e, u_{t}, s_{t}^a, m_{t}^a, b_{(s,m),t} ).  \\
    &\propto \sum_{j=1}^{|S|} \left[ \prb( o_{t+1} |  s_{t+1}^a = i, m_{t+1}^a = k, s_t^e, u_{t}, s_t^a, b_{(s,m),t} ) \right. \nonumber \\ 
    &~~~~~~~~~~~~~~~~\left. \prb( s_{t+1}^a = i, m_{t+1}^a = k | s_{t}^a=j, u_{t}, m_t^a = k ) \prb( s_{t}^a = j, m_{t}^a = k |  b_{(s,m),t} ) \right],   \\
    &\propto  \sum_{j=1}^{|S|} O(o_{t+1},s_{t+1}^a=i,s_t^a=j,m_t=k) T_s(s_{t+1}^a=i, s_{t}^a=j, m_t ) b_{(s,m),t}(s_t^a = j, m_{t}^a = k).
\end{align}
 


\newpage

The solution is to use a \textbackslash intertext\{ ... \} or \textbackslash shortintertext\{ ... \} block:

\begin{align}
     b_{(s,m),t+1}\left(\begin{matrix}
        s^a = i, \\ m^a = k
    \end{matrix} \right) &= \prb( s_{t+1}^a = i, m_{t+1}^a = k | o_{t+1}, s_{t}^e, u_{t}, s_t^a, m_t^a, b_{(s,m),t} ) \\
    &= \frac{\prb( o_{t+1} | s_{t+1}^a = i, m_{t+1}^a = k, s_{i,t}^e, u_{t},  s_t^a, m_t^a, b_{t} ) \prb( s_{t+1}^a = i, m_{t+1}^a = k |  s_{i,t}^e, u_{t},  s_t^a, m_t^a, b_{t} )}{\prb(o_{t+1} | s_{i,t}^e, u_{t}, s_t^a, m_t^a, b_{t}   ) } \\
    \shortintertext{ Comment explaining why the previous line is way to complicated and can be easily simplified } 
          &\propto \prb( o_{t+1} |  s_{t+1}^a = i, m_{t+1}^a = k, s_t^e, u_{t}, s_t^a, m_t^a, b_{t} ) \prb( s_{t+1}^a = i, m_{t+1}^a = k |  s_t^e, u_{t}, s_{t}^a, m_{t}^a, b_{(s,m),t} ).  \\
    &\propto \sum_{j=1}^{|S|} \left[ \prb( o_{t+1} |  s_{t+1}^a = i, m_{t+1}^a = k, s_t^e, u_{t}, s_t^a, b_{(s,m),t} ) \right. \nonumber \\ 
    &~~~~~~~~~~~~~~~~\left. \prb( s_{t+1}^a = i, m_{t+1}^a = k | s_{t}^a=j, u_{t}, m_t^a = k ) \prb( s_{t}^a = j, m_{t}^a = k |  b_{(s,m),t} ) \right],   \\
    &\propto  \sum_{j=1}^{|S|} O(o_{t+1},s_{t+1}^a=i,s_t^a=j,m_t=k) T_s(s_{t+1}^a=i, s_{t}^a=j, m_t ) b_{(s,m),t}(s_t^a = j, m_{t}^a = k).
\end{align}
 
\end{proof}

\newpage
\section{Citations!}
\theorem{Citation Placeholders}

Inserting a placeholder allows you to keep writing without breaking flow, while making a reminder to come back and resolve it later.

\begin{proof}
    Using the new command function, define
    \begin{lstlisting}
            \newcommand{\unk}[1][]{%
               \ifthenelse{ \equal{#1}{} }
                  {\red{$^{\emph{[Citation?]~}}$}}
                  {\red{$^{\emph{[#1?]~}}$}}
            }
    \end{lstlisting}

    Now, when composing, instead of coming to a stop and immediately looking up the necessary information, you can leave a place holder.  An optional argument allows notes to your future self.

    \\textit{}begin{quote}
        The quick brown fox \textbackslash unk jumped over the lazy dog \textbackslash unk[the cat]...
    \end{quote}

    becomes
    \begin{quote}
        The quick brown fox\unk jumped over the lazy dog\unk[the cat]...
    \end{quote}
    
\end{proof}

\newpage
\section{Proofing Copy}

\begin{theorem}[Proofing]
While the IEEE double column has a clean, professional look, it's often easier to markup double space.  Using a flag, and some header variables, switching between proofing and presentation is as easy as commenting out a line. 
\end{theorem}

\begin{proof}

    Make use of the \textbackslash \textit{newif} command to define a proof flag. Then at the start of your document set up two configurations.
    \begin{lstlisting}
    \newif\ifproofing\proofingtrue
    %\newif\ifproofing\proofingfalse
    
    \ifproofing
    %%%%%%%%%%%%%%%%%%%%%%%%%%%%%%%  proofing... %%%%%%%%%%%%%%%%%%%%%%%%%%%%%%%%%%%
    \documentclass{article}
    \usepackage[margin=1in]{geometry}
    \usepackage{setspace}
    \doublespacing
    %%%%%%%%%%%%%%%%%%%%%%%%%%%%%%%  proofing... %%%%%%%%%%%%%%%%%%%%%%%%%%%%%%%%%%%
    \else
    %%%%%%%%%%%%%%%%%%%%%%%%% comment out for proofing... %%%%%%%%%%%%%%%%%%%%%%%%%%
    \documentclass[journal]{IEEEtran}
    \IEEEoverridecommandlockouts                              % This command is only needed if 
                                                              % you want to use the \thanks command
    % \overrideIEEEmargins                                      % Needed to meet printer requirements.
    %%%%%%%%%%%%%%%%%%%%%%%%% comment out for proofing... %%%%%%%%%%%%%%%%%%%%%%%%%%
    \fi
    \end{lstlisting}

    \newpage
    You can also add comments that appear in the proofing document, but are removed automatically when you go to final copy.  

    \begin{lstlisting}
        \@ifundefined{ifproofing} 
        {%
            \newif\ifproofing
        }{%
            % \proofing defined
        }  
        
        \ifproofing
            \newcommand{\flag}[1]{\red{\vspace{4mm}\textsc{{#1}}\\\vspace{4mm}}}
        \else
            \newcommand{\flag}[1]{}
        \fi 
    \end{lstlisting}

\end{proof}


\section{Expectation, Probability}

The expectation of a random variable, X is $\E(x)$.  The probability of X, $\prb(X = 1)$

\newpage

\section{Adding a copyright notice on the first page}

When posting an IEEE submitted paper to an online archive such as \emph{Arxiv}, authors are supposed to add text to the first page indicating the copyright status.  This is the easiest method to add the text that I have found to date.

\begin{lstlisting}[language=TeX]
    \makeatletter
    \def\ps@IEEEtitlepagestyle{%
      \def\@oddfoot{\mycopyrightnotice}%
      \def\@oddhead{\hbox{}\@IEEEheaderstyle\leftmark\hfil\thepage}\relax
      \def\@evenhead{\@IEEEheaderstyle\thepage\hfil\leftmark\hbox{}}\relax
      \def\@evenfoot{}%
    }
    \def\mycopyrightnotice{%
      \begin{minipage}{\textwidth}
      \centering \scriptsize
      % Copyright~\copyright~20xx IEEE. Personal use of this material is permitted. Permission from IEEE must be obtained for all other uses, in any current or future media, including\\reprinting/republishing this material for advertising or promotional purposes, creating new collective works, for resale or redistribution to servers or lists, or reuse of any copyrighted component of this work in other works by sending a request to pubs-permissions@ieee.org.
        This work has been submitted to the IEEE for possible publication. Copyright may be transferred without notice, after which this version may no longer be accessible.  
      \end{minipage}
    }
    \makeatother
\end{lstlisting}


\bibliographystyle{IEEEtran}
\bibliography{bibliography}

\end{document}